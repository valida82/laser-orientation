% Options for packages loaded elsewhere
\PassOptionsToPackage{unicode}{hyperref}
\PassOptionsToPackage{hyphens}{url}
%
\documentclass[
]{article}
\usepackage{amsmath,amssymb}
\usepackage{lmodern}
\usepackage{iftex}
\ifPDFTeX
  \usepackage[T1]{fontenc}
  \usepackage[utf8]{inputenc}
  \usepackage{textcomp} % provide euro and other symbols
\else % if luatex or xetex
  \usepackage{unicode-math}
  \defaultfontfeatures{Scale=MatchLowercase}
  \defaultfontfeatures[\rmfamily]{Ligatures=TeX,Scale=1}
\fi
% Use upquote if available, for straight quotes in verbatim environments
\IfFileExists{upquote.sty}{\usepackage{upquote}}{}
\IfFileExists{microtype.sty}{% use microtype if available
  \usepackage[]{microtype}
  \UseMicrotypeSet[protrusion]{basicmath} % disable protrusion for tt fonts
}{}
\makeatletter
\@ifundefined{KOMAClassName}{% if non-KOMA class
  \IfFileExists{parskip.sty}{%
    \usepackage{parskip}
  }{% else
    \setlength{\parindent}{0pt}
    \setlength{\parskip}{6pt plus 2pt minus 1pt}}
}{% if KOMA class
  \KOMAoptions{parskip=half}}
\makeatother
\usepackage{xcolor}
\IfFileExists{xurl.sty}{\usepackage{xurl}}{} % add URL line breaks if available
\IfFileExists{bookmark.sty}{\usepackage{bookmark}}{\usepackage{hyperref}}
\hypersetup{
  pdftitle={LASER Badge},
  pdfauthor={YOUR NAME HERE},
  hidelinks,
  pdfcreator={LaTeX via pandoc}}
\urlstyle{same} % disable monospaced font for URLs
\usepackage[margin=1in]{geometry}
\usepackage{color}
\usepackage{fancyvrb}
\newcommand{\VerbBar}{|}
\newcommand{\VERB}{\Verb[commandchars=\\\{\}]}
\DefineVerbatimEnvironment{Highlighting}{Verbatim}{commandchars=\\\{\}}
% Add ',fontsize=\small' for more characters per line
\usepackage{framed}
\definecolor{shadecolor}{RGB}{248,248,248}
\newenvironment{Shaded}{\begin{snugshade}}{\end{snugshade}}
\newcommand{\AlertTok}[1]{\textcolor[rgb]{0.94,0.16,0.16}{#1}}
\newcommand{\AnnotationTok}[1]{\textcolor[rgb]{0.56,0.35,0.01}{\textbf{\textit{#1}}}}
\newcommand{\AttributeTok}[1]{\textcolor[rgb]{0.77,0.63,0.00}{#1}}
\newcommand{\BaseNTok}[1]{\textcolor[rgb]{0.00,0.00,0.81}{#1}}
\newcommand{\BuiltInTok}[1]{#1}
\newcommand{\CharTok}[1]{\textcolor[rgb]{0.31,0.60,0.02}{#1}}
\newcommand{\CommentTok}[1]{\textcolor[rgb]{0.56,0.35,0.01}{\textit{#1}}}
\newcommand{\CommentVarTok}[1]{\textcolor[rgb]{0.56,0.35,0.01}{\textbf{\textit{#1}}}}
\newcommand{\ConstantTok}[1]{\textcolor[rgb]{0.00,0.00,0.00}{#1}}
\newcommand{\ControlFlowTok}[1]{\textcolor[rgb]{0.13,0.29,0.53}{\textbf{#1}}}
\newcommand{\DataTypeTok}[1]{\textcolor[rgb]{0.13,0.29,0.53}{#1}}
\newcommand{\DecValTok}[1]{\textcolor[rgb]{0.00,0.00,0.81}{#1}}
\newcommand{\DocumentationTok}[1]{\textcolor[rgb]{0.56,0.35,0.01}{\textbf{\textit{#1}}}}
\newcommand{\ErrorTok}[1]{\textcolor[rgb]{0.64,0.00,0.00}{\textbf{#1}}}
\newcommand{\ExtensionTok}[1]{#1}
\newcommand{\FloatTok}[1]{\textcolor[rgb]{0.00,0.00,0.81}{#1}}
\newcommand{\FunctionTok}[1]{\textcolor[rgb]{0.00,0.00,0.00}{#1}}
\newcommand{\ImportTok}[1]{#1}
\newcommand{\InformationTok}[1]{\textcolor[rgb]{0.56,0.35,0.01}{\textbf{\textit{#1}}}}
\newcommand{\KeywordTok}[1]{\textcolor[rgb]{0.13,0.29,0.53}{\textbf{#1}}}
\newcommand{\NormalTok}[1]{#1}
\newcommand{\OperatorTok}[1]{\textcolor[rgb]{0.81,0.36,0.00}{\textbf{#1}}}
\newcommand{\OtherTok}[1]{\textcolor[rgb]{0.56,0.35,0.01}{#1}}
\newcommand{\PreprocessorTok}[1]{\textcolor[rgb]{0.56,0.35,0.01}{\textit{#1}}}
\newcommand{\RegionMarkerTok}[1]{#1}
\newcommand{\SpecialCharTok}[1]{\textcolor[rgb]{0.00,0.00,0.00}{#1}}
\newcommand{\SpecialStringTok}[1]{\textcolor[rgb]{0.31,0.60,0.02}{#1}}
\newcommand{\StringTok}[1]{\textcolor[rgb]{0.31,0.60,0.02}{#1}}
\newcommand{\VariableTok}[1]{\textcolor[rgb]{0.00,0.00,0.00}{#1}}
\newcommand{\VerbatimStringTok}[1]{\textcolor[rgb]{0.31,0.60,0.02}{#1}}
\newcommand{\WarningTok}[1]{\textcolor[rgb]{0.56,0.35,0.01}{\textbf{\textit{#1}}}}
\usepackage{graphicx}
\makeatletter
\def\maxwidth{\ifdim\Gin@nat@width>\linewidth\linewidth\else\Gin@nat@width\fi}
\def\maxheight{\ifdim\Gin@nat@height>\textheight\textheight\else\Gin@nat@height\fi}
\makeatother
% Scale images if necessary, so that they will not overflow the page
% margins by default, and it is still possible to overwrite the defaults
% using explicit options in \includegraphics[width, height, ...]{}
\setkeys{Gin}{width=\maxwidth,height=\maxheight,keepaspectratio}
% Set default figure placement to htbp
\makeatletter
\def\fps@figure{htbp}
\makeatother
\setlength{\emergencystretch}{3em} % prevent overfull lines
\providecommand{\tightlist}{%
  \setlength{\itemsep}{0pt}\setlength{\parskip}{0pt}}
\setcounter{secnumdepth}{-\maxdimen} % remove section numbering
\ifLuaTeX
  \usepackage{selnolig}  % disable illegal ligatures
\fi

\title{LASER Badge}
\usepackage{etoolbox}
\makeatletter
\providecommand{\subtitle}[1]{% add subtitle to \maketitle
  \apptocmd{\@title}{\par {\large #1 \par}}{}{}
}
\makeatother
\subtitle{LASER Institute Orientation}
\author{YOUR NAME HERE}
\date{July 12, 2022}

\begin{document}
\maketitle

{
\setcounter{tocdepth}{4}
\tableofcontents
}
\hypertarget{introduction}{%
\subsection{Introduction}\label{introduction}}

Welcome to your first LASER badge! This LASER Orientation Badge is
really a warm-up activitiy to introduce you to R Markdown and the coding
case studies that we will be using in the machine learning, network
analysis, and text mining labs. It is a chance to become familiar with
how RStudio and R Markdown works.

You may have used R before-or you may not have! Either is fine as this
task will be designed with the assumption that you have not used R
before. It includes ``reaches'' for anyone who may want to do a bit
more.

In the context of doing so, we'll focus on the following tasks:

\begin{enumerate}
\def\labelenumi{\arabic{enumi}.}
\tightlist
\item
  Reading data into R (in the \textbf{Prepare} section)
\item
  Preparing and ``wrangling'' data in table (think spreadsheet!) format
  (in the \textbf{Wrangle} section)
\item
  Creating some plots (in the \textbf{Explore} section)
\item
  Running a model - specifically, a regression model (in the
  \textbf{Model} section)
\item
  Finally, creating a reproducible report of your work you can share
  with others (in the \textbf{Communicate} section)
\end{enumerate}

\hypertarget{the-laser-cycle}{%
\subsubsection{The LASER Cycle}\label{the-laser-cycle}}

You may be wondering what these bolded terms above refer to; what's so
special about preparing, wrangling, exploring, and modeling data - and
communicating results? We're using these terms as a part of a framework,
or model, for what we mean by doing learning in STEM education research.

The particular framework we are using comes from the work of Krumm et
al.'s
\href{https://github.com/laser-institute/essential-readings/blob/main/laser-orientation/Learning\%20Analytics\%20Goes\%20to\%20School.pdf}{\emph{Learning
Analytics Goes to School}}\emph{.} You can check that out, but don't
feel any need to dive deep for now - we'll be spending more time on this
in first day of the summer institute. For now, know that this document
is organized around three of the five components of what we're referring
to as the \textbf{LASER cycle}.

Click the green arrow to the right of the ``code chunk'' below to view
the image (more on that process of clicking the green arrow and what it
does, too, in a moment)!

\begin{Shaded}
\begin{Highlighting}[]
\NormalTok{knitr}\SpecialCharTok{::}\FunctionTok{include\_graphics}\NormalTok{(}\StringTok{"img/laser{-}cycle.png"}\NormalTok{)}
\end{Highlighting}
\end{Shaded}

\includegraphics{img/laser-cycle.png}

\hypertarget{how-to-use-this-r-markdown-document}{%
\subsubsection{How to use this R Markdown
document}\label{how-to-use-this-r-markdown-document}}

This is an R Markdown file as indicated by the .rmd extension at the end
of the file name. R Markdown documents are fully reproducible and use a
productive
\href{https://bookdown.org/yihui/rmarkdown/notebook.html}{notebook
interface} to combine narrative text and ``chunks'' of code to produce a
range of formatted outputs including: formats
including~\href{https://bookdown.org/yihui/rmarkdown/html-document.html}{HTML},~\href{https://bookdown.org/yihui/rmarkdown/pdf-document.html}{PDF},~\href{https://bookdown.org/yihui/rmarkdown/word-document.html}{MS
Word},~\href{https://bookdown.org/yihui/rmarkdown/beamer-presentation.html}{Beamer},~\href{https://bookdown.org/yihui/rmarkdown/ioslides-presentation.html}{HTML5
slides},~\href{https://bookdown.org/yihui/rmarkdown/tufte-handouts.html}{Tufte-style
handouts},~\href{https://bookdown.org/}{books},~\href{https://rmarkdown.rstudio.com/flexdashboard/}{dashboards},~\href{https://bookdown.org/yihui/rmarkdown/shiny-documents.html}{shiny
applications},~\href{https://github.com/rstudio/rticles}{scientific
articles},~\href{https://bookdown.org/yihui/rmarkdown/rmarkdown-site.html}{websites},
and more.

There are two keys to your use of R Markdown for this activity:

\begin{enumerate}
\def\labelenumi{\arabic{enumi}.}
\tightlist
\item
  First, be sure that you are viewing the document in the ``Visual
  Editor'' mode. You can use this mode by clicking the word ``Visual''
  on the left side of the toolbar above.
\item
  Second, click ``Knit'' next to the yarn ball at the top of this screen
  to preview the document as you work through it. This will allow you to
  see your code and the input in a rendered - easy-to-read - document,
  just as others will see this document when shared. Try knitting the
  document now and see what happens.
\end{enumerate}

Let's get started! We are glad you are here and to begin this exciting
(and challenging) journey together.

\hypertarget{prepare}{%
\subsection{1. PREPARE}\label{prepare}}

By preparing, we refer to developing a question or purpose for the
analysis, which you likely know from your research can be difficult!
This part of the process also involves developing an understanding of
the data and what you may need to analyze the data. This often involves
looking at the data and its documentation. For now, we'll focus on just
a few parts of this process, diving in much more deeply over the coming
weeks.

\hypertarget{packages}{%
\subsubsection{Packages 📦}\label{packages}}

R uses ``packages,'' add-ons that enhance its functionality. One package
that we'll be using is the tidyverse. To load the tidyverse, click the
green arrow in the right corner of the block-or ``chunk''-of code that
follows.

\begin{Shaded}
\begin{Highlighting}[]
\FunctionTok{library}\NormalTok{(tidyverse)}
\end{Highlighting}
\end{Shaded}

\begin{verbatim}
## -- Attaching packages --------------------------------------- tidyverse 1.3.1 --
\end{verbatim}

\begin{verbatim}
## v ggplot2 3.3.6     v purrr   0.3.4
## v tibble  3.1.7     v dplyr   1.0.9
## v tidyr   1.2.0     v stringr 1.4.0
## v readr   2.1.2     v forcats 0.5.1
\end{verbatim}

\begin{verbatim}
## -- Conflicts ------------------------------------------ tidyverse_conflicts() --
## x dplyr::filter() masks stats::filter()
## x dplyr::lag()    masks stats::lag()
\end{verbatim}

Please do not worry if you saw a number of messages: those probably mean
that the tidyverse loaded just fine. If you see an error, though, try to
interpret or search via your search engine the contents of the error, or
reach out to us for assistance.

\hypertarget{loading-or-reading-in-data}{%
\subsubsection{Loading (or reading in)
data}\label{loading-or-reading-in-data}}

Next, we'll load data-specifically, a CSV file, the kind that you can
export from Microsoft Excel or Google Sheets - into R, using the
\texttt{read\_csv()} function in the next chunk.

Clicking the green arrow runs the code; do that next to read the
\texttt{sci-online-classes.csv} file stored in your data folder into
your R environment:

\begin{Shaded}
\begin{Highlighting}[]
\NormalTok{d }\OtherTok{\textless{}{-}} \FunctionTok{read\_csv}\NormalTok{(}\StringTok{"data/sci{-}online{-}classes.csv"}\NormalTok{)}
\end{Highlighting}
\end{Shaded}

\begin{verbatim}
## Rows: 603 Columns: 30
## -- Column specification --------------------------------------------------------
## Delimiter: ","
## chr  (6): course_id, subject, semester, section, Gradebook_Item, Gender
## dbl (23): student_id, total_points_possible, total_points_earned, percentage...
## lgl  (1): Grade_Category
## 
## i Use `spec()` to retrieve the full column specification for this data.
## i Specify the column types or set `show_col_types = FALSE` to quiet this message.
\end{verbatim}

Nice work! You should now see a new data ``object'' named \texttt{d}
saved in your Environment pane. Try clicking on it and see what happens.

\hypertarget{viewing-or-inspecting-data}{%
\paragraph{Viewing or inspecting
data}\label{viewing-or-inspecting-data}}

Now let's learn another way to inspect our data. Run the next chunk and
look at the results, tabbing left or right with the arrows, or scanning
through the rows by clicking the numbers at the bottom of the pane with
the print-out of the data you loaded:

\begin{Shaded}
\begin{Highlighting}[]
\NormalTok{d}
\end{Highlighting}
\end{Shaded}

\begin{verbatim}
## # A tibble: 603 x 30
##    student_id course_id     total_points_poss~ total_points_ea~ percentage_earn~
##         <dbl> <chr>                      <dbl>            <dbl>            <dbl>
##  1      43146 FrScA-S216-02               3280             2220            0.677
##  2      44638 OcnA-S116-01                3531             2672            0.757
##  3      47448 FrScA-S216-01               2870             1897            0.661
##  4      47979 OcnA-S216-01                4562             3090            0.677
##  5      48797 PhysA-S116-01               2207             1910            0.865
##  6      51943 FrScA-S216-03               4208             3596            0.855
##  7      52326 AnPhA-S216-01               4325             2255            0.521
##  8      52446 PhysA-S116-01               2086             1719            0.824
##  9      53447 FrScA-S116-01               4655             3149            0.676
## 10      53475 FrScA-S116-02               1710             1402            0.820
## # ... with 593 more rows, and 25 more variables: subject <chr>, semester <chr>,
## #   section <chr>, Gradebook_Item <chr>, Grade_Category <lgl>,
## #   FinalGradeCEMS <dbl>, Points_Possible <dbl>, Points_Earned <dbl>,
## #   Gender <chr>, q1 <dbl>, q2 <dbl>, q3 <dbl>, q4 <dbl>, q5 <dbl>, q6 <dbl>,
## #   q7 <dbl>, q8 <dbl>, q9 <dbl>, q10 <dbl>, TimeSpent <dbl>,
## #   TimeSpent_hours <dbl>, TimeSpent_std <dbl>, int <dbl>, pc <dbl>, uv <dbl>
\end{verbatim}

\hypertarget{your-turn}{%
\paragraph{\texorpdfstring{{\textbf{Your Turn}}
\textbf{⤵}}{Your Turn ⤵}}\label{your-turn}}

What do you notice about this data set? What do you wonder? Add one-two
thoughts following the dashes next (you can add additional dashes if you
like!):

\begin{itemize}
\tightlist
\item
\item
\end{itemize}

There are other ways to inspect your data; the \texttt{glimpse()}
function provides one such way. Run the code below to take a glimpse at
your data.

\begin{Shaded}
\begin{Highlighting}[]
\FunctionTok{glimpse}\NormalTok{(d)}
\end{Highlighting}
\end{Shaded}

\begin{verbatim}
## Rows: 603
## Columns: 30
## $ student_id            <dbl> 43146, 44638, 47448, 47979, 48797, 51943, 52326,~
## $ course_id             <chr> "FrScA-S216-02", "OcnA-S116-01", "FrScA-S216-01"~
## $ total_points_possible <dbl> 3280, 3531, 2870, 4562, 2207, 4208, 4325, 2086, ~
## $ total_points_earned   <dbl> 2220, 2672, 1897, 3090, 1910, 3596, 2255, 1719, ~
## $ percentage_earned     <dbl> 0.6768293, 0.7567261, 0.6609756, 0.6773345, 0.86~
## $ subject               <chr> "FrScA", "OcnA", "FrScA", "OcnA", "PhysA", "FrSc~
## $ semester              <chr> "S216", "S116", "S216", "S216", "S116", "S216", ~
## $ section               <chr> "02", "01", "01", "01", "01", "03", "01", "01", ~
## $ Gradebook_Item        <chr> "POINTS EARNED & TOTAL COURSE POINTS", "ATTEMPTE~
## $ Grade_Category        <lgl> NA, NA, NA, NA, NA, NA, NA, NA, NA, NA, NA, NA, ~
## $ FinalGradeCEMS        <dbl> 93.45372, 81.70184, 88.48758, 81.85260, 84.00000~
## $ Points_Possible       <dbl> 5, 10, 10, 5, 438, 5, 10, 10, 443, 5, 12, 10, 5,~
## $ Points_Earned         <dbl> NA, 10.00, NA, 4.00, 399.00, NA, NA, 10.00, 425.~
## $ Gender                <chr> "M", "F", "M", "M", "F", "F", "M", "F", "F", "M"~
## $ q1                    <dbl> 5, 4, 5, 5, 4, NA, 5, 3, 4, NA, NA, 4, 3, 5, NA,~
## $ q2                    <dbl> 4, 4, 4, 5, 3, NA, 5, 3, 3, NA, NA, 5, 3, 3, NA,~
## $ q3                    <dbl> 4, 3, 4, 3, 3, NA, 3, 3, 3, NA, NA, 3, 3, 5, NA,~
## $ q4                    <dbl> 5, 4, 5, 5, 4, NA, 5, 3, 4, NA, NA, 5, 3, 5, NA,~
## $ q5                    <dbl> 5, 4, 5, 5, 4, NA, 5, 3, 4, NA, NA, 5, 4, 5, NA,~
## $ q6                    <dbl> 5, 4, 4, 5, 4, NA, 5, 4, 3, NA, NA, 5, 3, 5, NA,~
## $ q7                    <dbl> 5, 4, 4, 4, 4, NA, 4, 3, 3, NA, NA, 5, 3, 5, NA,~
## $ q8                    <dbl> 5, 5, 5, 5, 4, NA, 5, 3, 4, NA, NA, 4, 3, 5, NA,~
## $ q9                    <dbl> 4, 4, 3, 5, NA, NA, 5, 3, 2, NA, NA, 5, 2, 2, NA~
## $ q10                   <dbl> 5, 4, 5, 5, 3, NA, 5, 3, 5, NA, NA, 4, 4, 5, NA,~
## $ TimeSpent             <dbl> 1555.1667, 1382.7001, 860.4335, 1598.6166, 1481.~
## $ TimeSpent_hours       <dbl> 25.91944500, 23.04500167, 14.34055833, 26.643610~
## $ TimeSpent_std         <dbl> -0.18051496, -0.30780313, -0.69325954, -0.148446~
## $ int                   <dbl> 5.0, 4.2, 5.0, 5.0, 3.8, 4.6, 5.0, 3.0, 4.2, NA,~
## $ pc                    <dbl> 4.50, 3.50, 4.00, 3.50, 3.50, 4.00, 3.50, 3.00, ~
## $ uv                    <dbl> 4.333333, 4.000000, 3.666667, 5.000000, 3.500000~
\end{verbatim}

We have one more question to pose to you: What do rows and columns
typically represent in your area of work and/or research?

Generally, rows typically represent ``cases,'' the units that we
measure, or the units on which we collect data. This is not a trick
question! What counts as a ``case'' (and therefore what is represented
as a row) varies by (and within) fields. There may be multiple types or
levels of units studied in your field; listing more than one is fine!
Also, please consider what columns - which usually represent variables -
represent in your area of work and/or research.

\hypertarget{your-turn-1}{%
\paragraph{\texorpdfstring{{\textbf{Your Turn}}
\textbf{⤵}}{Your Turn ⤵}}\label{your-turn-1}}

What rows typically (or you think may) represent:

\begin{itemize}
\tightlist
\item
\end{itemize}

What columns typically (or you think may) represent:

\begin{itemize}
\tightlist
\item
\end{itemize}

Next, we'll use a few functions that are handy for preparing data in
table form.

\hypertarget{wrangle}{%
\subsection{2. WRANGLE}\label{wrangle}}

By wrangle, we refer to the process of cleaning and processing data,
and, in cases, merging (or joining) data from multiple sources. Often,
this part of the process is very (surprisingly) time-intensive.
Wrangling your data into shape can itself be an important
accomplishment! There are great tools in R to do this, especially
through the use of the \{dplyr\} R package.

\hypertarget{selecting-variables}{%
\subsubsection{Selecting variables}\label{selecting-variables}}

Let's select only a few variables.

\begin{Shaded}
\begin{Highlighting}[]
\NormalTok{d }\SpecialCharTok{|}\ErrorTok{\textgreater{}} 
  \FunctionTok{select}\NormalTok{(student\_id, total\_points\_possible, total\_points\_earned)}
\end{Highlighting}
\end{Shaded}

\begin{verbatim}
## # A tibble: 603 x 3
##    student_id total_points_possible total_points_earned
##         <dbl>                 <dbl>               <dbl>
##  1      43146                  3280                2220
##  2      44638                  3531                2672
##  3      47448                  2870                1897
##  4      47979                  4562                3090
##  5      48797                  2207                1910
##  6      51943                  4208                3596
##  7      52326                  4325                2255
##  8      52446                  2086                1719
##  9      53447                  4655                3149
## 10      53475                  1710                1402
## # ... with 593 more rows
\end{verbatim}

Notice how the number of columns (variables) is now different.

Let's \emph{include one additional variable} in your select function.

First, we need to figure out what variables exist in our dataset (or be
reminded of this - it's very common in R to be continually checking and
inspecting your data)!

You can use a function named glimpse() to do this.

\begin{Shaded}
\begin{Highlighting}[]
\FunctionTok{glimpse}\NormalTok{(d)}
\end{Highlighting}
\end{Shaded}

\begin{verbatim}
## Rows: 603
## Columns: 30
## $ student_id            <dbl> 43146, 44638, 47448, 47979, 48797, 51943, 52326,~
## $ course_id             <chr> "FrScA-S216-02", "OcnA-S116-01", "FrScA-S216-01"~
## $ total_points_possible <dbl> 3280, 3531, 2870, 4562, 2207, 4208, 4325, 2086, ~
## $ total_points_earned   <dbl> 2220, 2672, 1897, 3090, 1910, 3596, 2255, 1719, ~
## $ percentage_earned     <dbl> 0.6768293, 0.7567261, 0.6609756, 0.6773345, 0.86~
## $ subject               <chr> "FrScA", "OcnA", "FrScA", "OcnA", "PhysA", "FrSc~
## $ semester              <chr> "S216", "S116", "S216", "S216", "S116", "S216", ~
## $ section               <chr> "02", "01", "01", "01", "01", "03", "01", "01", ~
## $ Gradebook_Item        <chr> "POINTS EARNED & TOTAL COURSE POINTS", "ATTEMPTE~
## $ Grade_Category        <lgl> NA, NA, NA, NA, NA, NA, NA, NA, NA, NA, NA, NA, ~
## $ FinalGradeCEMS        <dbl> 93.45372, 81.70184, 88.48758, 81.85260, 84.00000~
## $ Points_Possible       <dbl> 5, 10, 10, 5, 438, 5, 10, 10, 443, 5, 12, 10, 5,~
## $ Points_Earned         <dbl> NA, 10.00, NA, 4.00, 399.00, NA, NA, 10.00, 425.~
## $ Gender                <chr> "M", "F", "M", "M", "F", "F", "M", "F", "F", "M"~
## $ q1                    <dbl> 5, 4, 5, 5, 4, NA, 5, 3, 4, NA, NA, 4, 3, 5, NA,~
## $ q2                    <dbl> 4, 4, 4, 5, 3, NA, 5, 3, 3, NA, NA, 5, 3, 3, NA,~
## $ q3                    <dbl> 4, 3, 4, 3, 3, NA, 3, 3, 3, NA, NA, 3, 3, 5, NA,~
## $ q4                    <dbl> 5, 4, 5, 5, 4, NA, 5, 3, 4, NA, NA, 5, 3, 5, NA,~
## $ q5                    <dbl> 5, 4, 5, 5, 4, NA, 5, 3, 4, NA, NA, 5, 4, 5, NA,~
## $ q6                    <dbl> 5, 4, 4, 5, 4, NA, 5, 4, 3, NA, NA, 5, 3, 5, NA,~
## $ q7                    <dbl> 5, 4, 4, 4, 4, NA, 4, 3, 3, NA, NA, 5, 3, 5, NA,~
## $ q8                    <dbl> 5, 5, 5, 5, 4, NA, 5, 3, 4, NA, NA, 4, 3, 5, NA,~
## $ q9                    <dbl> 4, 4, 3, 5, NA, NA, 5, 3, 2, NA, NA, 5, 2, 2, NA~
## $ q10                   <dbl> 5, 4, 5, 5, 3, NA, 5, 3, 5, NA, NA, 4, 4, 5, NA,~
## $ TimeSpent             <dbl> 1555.1667, 1382.7001, 860.4335, 1598.6166, 1481.~
## $ TimeSpent_hours       <dbl> 25.91944500, 23.04500167, 14.34055833, 26.643610~
## $ TimeSpent_std         <dbl> -0.18051496, -0.30780313, -0.69325954, -0.148446~
## $ int                   <dbl> 5.0, 4.2, 5.0, 5.0, 3.8, 4.6, 5.0, 3.0, 4.2, NA,~
## $ pc                    <dbl> 4.50, 3.50, 4.00, 3.50, 3.50, 4.00, 3.50, 3.00, ~
## $ uv                    <dbl> 4.333333, 4.000000, 3.666667, 5.000000, 3.500000~
\end{verbatim}

\hypertarget{your-turn-2}{%
\paragraph{\texorpdfstring{{\textbf{Your Turn}}
\textbf{⤵}}{Your Turn ⤵}}\label{your-turn-2}}

In the code chunk below, add a new variable to the code below, being
careful to type the new variable name as it appears in the data. We've
added some code to get you started. Consider how the names of the other
variables are separated as you think about how to add an additional
variable to this code.

\begin{Shaded}
\begin{Highlighting}[]
\NormalTok{d }\SpecialCharTok{|}\ErrorTok{\textgreater{}} 
  \FunctionTok{select}\NormalTok{(student\_id, total\_points\_possible, total\_points\_earned)}
\end{Highlighting}
\end{Shaded}

\begin{verbatim}
## # A tibble: 603 x 3
##    student_id total_points_possible total_points_earned
##         <dbl>                 <dbl>               <dbl>
##  1      43146                  3280                2220
##  2      44638                  3531                2672
##  3      47448                  2870                1897
##  4      47979                  4562                3090
##  5      48797                  2207                1910
##  6      51943                  4208                3596
##  7      52326                  4325                2255
##  8      52446                  2086                1719
##  9      53447                  4655                3149
## 10      53475                  1710                1402
## # ... with 593 more rows
\end{verbatim}

Once added, the output should be different than in the code above -
there should now be an additional variable included in the print-out.

\hypertarget{filtering-variables}{%
\subsubsection{Filtering variables}\label{filtering-variables}}

Next, let's explore filtering variables. Check out and run the next
chunk of code, imagining that we wish to filter our data to view only
the rows associated with students who earned a final grade (as a
percentage) of 70 - 70\% - or higher.

\begin{Shaded}
\begin{Highlighting}[]
\NormalTok{d }\SpecialCharTok{|}\ErrorTok{\textgreater{}} 
  \FunctionTok{filter}\NormalTok{(FinalGradeCEMS }\SpecialCharTok{\textgreater{}} \DecValTok{70}\NormalTok{)}
\end{Highlighting}
\end{Shaded}

\begin{verbatim}
## # A tibble: 438 x 30
##    student_id course_id     total_points_poss~ total_points_ea~ percentage_earn~
##         <dbl> <chr>                      <dbl>            <dbl>            <dbl>
##  1      43146 FrScA-S216-02               3280             2220            0.677
##  2      44638 OcnA-S116-01                3531             2672            0.757
##  3      47448 FrScA-S216-01               2870             1897            0.661
##  4      47979 OcnA-S216-01                4562             3090            0.677
##  5      48797 PhysA-S116-01               2207             1910            0.865
##  6      52326 AnPhA-S216-01               4325             2255            0.521
##  7      52446 PhysA-S116-01               2086             1719            0.824
##  8      53447 FrScA-S116-01               4655             3149            0.676
##  9      53475 FrScA-S216-01               1209              977            0.808
## 10      54066 OcnA-S116-01                4641             3429            0.739
## # ... with 428 more rows, and 25 more variables: subject <chr>, semester <chr>,
## #   section <chr>, Gradebook_Item <chr>, Grade_Category <lgl>,
## #   FinalGradeCEMS <dbl>, Points_Possible <dbl>, Points_Earned <dbl>,
## #   Gender <chr>, q1 <dbl>, q2 <dbl>, q3 <dbl>, q4 <dbl>, q5 <dbl>, q6 <dbl>,
## #   q7 <dbl>, q8 <dbl>, q9 <dbl>, q10 <dbl>, TimeSpent <dbl>,
## #   TimeSpent_hours <dbl>, TimeSpent_std <dbl>, int <dbl>, pc <dbl>, uv <dbl>
\end{verbatim}

\hypertarget{your-turn-3}{%
\subparagraph{\texorpdfstring{{\textbf{Your Turn}}
\textbf{⤵}}{Your Turn ⤵}}\label{your-turn-3}}

In the next code chunk, change the cut-off from 70\% to some other value
- larger or smaller (maybe much larger or smaller - feel free to play
around with the code a bit!).

\begin{Shaded}
\begin{Highlighting}[]
\NormalTok{d }\SpecialCharTok{|}\ErrorTok{\textgreater{}} 
  \FunctionTok{filter}\NormalTok{(FinalGradeCEMS }\SpecialCharTok{\textgreater{}} \DecValTok{70}\NormalTok{)}
\end{Highlighting}
\end{Shaded}

\begin{verbatim}
## # A tibble: 438 x 30
##    student_id course_id     total_points_poss~ total_points_ea~ percentage_earn~
##         <dbl> <chr>                      <dbl>            <dbl>            <dbl>
##  1      43146 FrScA-S216-02               3280             2220            0.677
##  2      44638 OcnA-S116-01                3531             2672            0.757
##  3      47448 FrScA-S216-01               2870             1897            0.661
##  4      47979 OcnA-S216-01                4562             3090            0.677
##  5      48797 PhysA-S116-01               2207             1910            0.865
##  6      52326 AnPhA-S216-01               4325             2255            0.521
##  7      52446 PhysA-S116-01               2086             1719            0.824
##  8      53447 FrScA-S116-01               4655             3149            0.676
##  9      53475 FrScA-S216-01               1209              977            0.808
## 10      54066 OcnA-S116-01                4641             3429            0.739
## # ... with 428 more rows, and 25 more variables: subject <chr>, semester <chr>,
## #   section <chr>, Gradebook_Item <chr>, Grade_Category <lgl>,
## #   FinalGradeCEMS <dbl>, Points_Possible <dbl>, Points_Earned <dbl>,
## #   Gender <chr>, q1 <dbl>, q2 <dbl>, q3 <dbl>, q4 <dbl>, q5 <dbl>, q6 <dbl>,
## #   q7 <dbl>, q8 <dbl>, q9 <dbl>, q10 <dbl>, TimeSpent <dbl>,
## #   TimeSpent_hours <dbl>, TimeSpent_std <dbl>, int <dbl>, pc <dbl>, uv <dbl>
\end{verbatim}

What happens when you change the cut-off from 70 to something else? Add
a thought (or more):

\begin{itemize}
\tightlist
\item
\end{itemize}

\hypertarget{arrange}{%
\subsubsection{Arrange}\label{arrange}}

The last function we'll use for preparing tables is arrange.

We'll combine this arrange() function with a function we used already -
select(). We do this so we can view only the student ID and their final
grade.

\begin{Shaded}
\begin{Highlighting}[]
\NormalTok{d }\SpecialCharTok{|}\ErrorTok{\textgreater{}} 
  \FunctionTok{select}\NormalTok{(student\_id, FinalGradeCEMS) }\SpecialCharTok{|}\ErrorTok{\textgreater{}} 
  \FunctionTok{arrange}\NormalTok{(FinalGradeCEMS)}
\end{Highlighting}
\end{Shaded}

\begin{verbatim}
## # A tibble: 603 x 2
##    student_id FinalGradeCEMS
##         <dbl>          <dbl>
##  1      90995          0    
##  2      92606          0.535
##  3      95684          0.903
##  4      90996          1.80 
##  5      94876          2.93 
##  6      92633          3.01 
##  7      85390          3.06 
##  8      94630          3.43 
##  9      90995          5.04 
## 10      96677          5.2  
## # ... with 593 more rows
\end{verbatim}

Note that arrange works by sorting values in ascending order (from
lowest to highest); you can change this by using the desc() function
with arrange, like the following:

\begin{Shaded}
\begin{Highlighting}[]
\NormalTok{d }\SpecialCharTok{|}\ErrorTok{\textgreater{}} 
  \FunctionTok{select}\NormalTok{(student\_id, FinalGradeCEMS) }\SpecialCharTok{|}\ErrorTok{\textgreater{}} 
  \FunctionTok{arrange}\NormalTok{(}\FunctionTok{desc}\NormalTok{(FinalGradeCEMS))}
\end{Highlighting}
\end{Shaded}

\begin{verbatim}
## # A tibble: 603 x 2
##    student_id FinalGradeCEMS
##         <dbl>          <dbl>
##  1      85650          100  
##  2      91067           99.8
##  3      66740           99.3
##  4      86792           99.1
##  5      78153           99.0
##  6      66689           98.6
##  7      88261           98.6
##  8      92740           98.6
##  9      92726           98.2
## 10      92741           98.2
## # ... with 593 more rows
\end{verbatim}

\hypertarget{your-turn-4}{%
\paragraph{\texorpdfstring{{\textbf{Your Turn}}
\textbf{⤵}}{Your Turn ⤵}}\label{your-turn-4}}

In the code chunk below, replace FinalGradeCEMS that is used with both
the select() and arrange() functions with a different variable in the
data set. Consider returning to the code chunk above in which you
glimpsed at the names of all of the variables.

\begin{Shaded}
\begin{Highlighting}[]
\NormalTok{d }\SpecialCharTok{|}\ErrorTok{\textgreater{}} 
  \FunctionTok{select}\NormalTok{(student\_id, FinalGradeCEMS) }\SpecialCharTok{|}\ErrorTok{\textgreater{}} 
  \FunctionTok{arrange}\NormalTok{(}\FunctionTok{desc}\NormalTok{(FinalGradeCEMS))}
\end{Highlighting}
\end{Shaded}

\begin{verbatim}
## # A tibble: 603 x 2
##    student_id FinalGradeCEMS
##         <dbl>          <dbl>
##  1      85650          100  
##  2      91067           99.8
##  3      66740           99.3
##  4      86792           99.1
##  5      78153           99.0
##  6      66689           98.6
##  7      88261           98.6
##  8      92740           98.6
##  9      92726           98.2
## 10      92741           98.2
## # ... with 593 more rows
\end{verbatim}

\hypertarget{reach-1}{%
\subsubsection{Reach 1 🎉}\label{reach-1}}

Can you compose a series of functions that include the select(),
filter(), and arrange functions? Recall that you can ``pipe'' the output
from one function to the next as when we used select() and arrange()
together in the code chunk above.

\emph{This reach is not required/necessary to complete; it's just for
those who wish to do a bit more with these functions at this time (we'll
do more in class, too!)}

\hypertarget{explore}{%
\subsection{3. EXPLORE}\label{explore}}

Exploratory data analysis, or exploring your data, involves processes of
\emph{describing} your data (such as by calculating the means and
standard deviations of numeric variables, or counting the frequency of
categorical variables) and, often, visualizing your data prior. In this
section, we'll create a few plots to explore our data.

\hypertarget{histogram}{%
\subsubsection{Histogram}\label{histogram}}

The code below creates a histogram, or a distribution of the values, in
this case for students' final grades.

\begin{Shaded}
\begin{Highlighting}[]
\FunctionTok{ggplot}\NormalTok{(d, }\FunctionTok{aes}\NormalTok{(}\AttributeTok{x =}\NormalTok{ FinalGradeCEMS)) }\SpecialCharTok{+}
  \FunctionTok{geom\_histogram}\NormalTok{()}
\end{Highlighting}
\end{Shaded}

\begin{verbatim}
## `stat_bin()` using `bins = 30`. Pick better value with `binwidth`.
\end{verbatim}

\begin{verbatim}
## Warning: Removed 30 rows containing non-finite values (stat_bin).
\end{verbatim}

\includegraphics{laser-orientation-badge_files/figure-latex/unnamed-chunk-15-1.pdf}

You can change the color of the histogram bars by specifying a color as
follows:

\begin{Shaded}
\begin{Highlighting}[]
\FunctionTok{ggplot}\NormalTok{(d, }\FunctionTok{aes}\NormalTok{(}\AttributeTok{x =}\NormalTok{ FinalGradeCEMS)) }\SpecialCharTok{+}
  \FunctionTok{geom\_histogram}\NormalTok{(}\AttributeTok{fill =} \StringTok{"blue"}\NormalTok{)}
\end{Highlighting}
\end{Shaded}

\begin{verbatim}
## `stat_bin()` using `bins = 30`. Pick better value with `binwidth`.
\end{verbatim}

\begin{verbatim}
## Warning: Removed 30 rows containing non-finite values (stat_bin).
\end{verbatim}

\includegraphics{laser-orientation-badge_files/figure-latex/unnamed-chunk-16-1.pdf}

\hypertarget{changing-colors}{%
\subsubsection{Changing colors}\label{changing-colors}}

\hypertarget{your-turn-5}{%
\paragraph{\texorpdfstring{{\textbf{Your Turn}}
\textbf{⤵}}{Your Turn ⤵}}\label{your-turn-5}}

In the code chunk below, change the color to one of your choosing;
consider this list of valid color names here:
\url{http://www.stat.columbia.edu/~tzheng/files/Rcolor.pdf}

\begin{Shaded}
\begin{Highlighting}[]
\FunctionTok{ggplot}\NormalTok{(d, }\FunctionTok{aes}\NormalTok{(}\AttributeTok{x =}\NormalTok{ FinalGradeCEMS)) }\SpecialCharTok{+}
  \FunctionTok{geom\_histogram}\NormalTok{(}\AttributeTok{fill =} \StringTok{"blue"}\NormalTok{)}
\end{Highlighting}
\end{Shaded}

\begin{verbatim}
## `stat_bin()` using `bins = 30`. Pick better value with `binwidth`.
\end{verbatim}

\begin{verbatim}
## Warning: Removed 30 rows containing non-finite values (stat_bin).
\end{verbatim}

\includegraphics{laser-orientation-badge_files/figure-latex/unnamed-chunk-17-1.pdf}

Finally, we'll make one more change; visualize the distribution of
another variable in the data - one other than FinalGradeCEMS. You can do
so by swapping out the name for another variable with FinalGradeCEMS.
Also, change the color to one other than blue.

\begin{Shaded}
\begin{Highlighting}[]
\FunctionTok{ggplot}\NormalTok{(d, }\FunctionTok{aes}\NormalTok{(}\AttributeTok{x =}\NormalTok{ FinalGradeCEMS)) }\SpecialCharTok{+}
  \FunctionTok{geom\_histogram}\NormalTok{(}\AttributeTok{fill =} \StringTok{"blue"}\NormalTok{)}
\end{Highlighting}
\end{Shaded}

\begin{verbatim}
## `stat_bin()` using `bins = 30`. Pick better value with `binwidth`.
\end{verbatim}

\begin{verbatim}
## Warning: Removed 30 rows containing non-finite values (stat_bin).
\end{verbatim}

\includegraphics{laser-orientation-badge_files/figure-latex/unnamed-chunk-18-1.pdf}

\hypertarget{reach-2}{%
\subsubsection{Reach 2 🎉}\label{reach-2}}

Completed the above? Nice job! Try for a ``reach'' by creating a scatter
plot for the relationship between two variables. You will need to pass
the names of two variables to the code below for what is now simply XXX
(a placeholder).

\begin{Shaded}
\begin{Highlighting}[]
\FunctionTok{ggplot}\NormalTok{(d, }\FunctionTok{aes}\NormalTok{(}\AttributeTok{x =}\NormalTok{ XXX, }\AttributeTok{y =}\NormalTok{ XXX)) }\SpecialCharTok{+}
  \FunctionTok{geom\_point}\NormalTok{()}
\end{Highlighting}
\end{Shaded}

\begin{verbatim}
## Error in FUN(X[[i]], ...): object 'XXX' not found
\end{verbatim}

\includegraphics{laser-orientation-badge_files/figure-latex/unnamed-chunk-19-1.pdf}

\hypertarget{model}{%
\subsection{4. MODEL}\label{model}}

``Model'' is one of those terms that has many different meanings. For
our purpose, we refer to the process of simplifying and summarizing our
data. Thus, models can take many forms; calculating means represents a
legitimate form of modeling data, as does estimating more complex
models, including linear regressions, and models and algorithms
associated with machine learning tasks. For now, we'll run a linear
regression to predict students' final grades.

Below, we predict students' final grades (\texttt{FinaGradeCEMS}, which
is on a 0-100 point scale) on the basis of the time they spent on the
course (measured through their learning management system in minutes,
\texttt{TimeSpent}, and the subject (one of five) of their specific
course.

\begin{Shaded}
\begin{Highlighting}[]
\NormalTok{m1 }\OtherTok{\textless{}{-}} \FunctionTok{lm}\NormalTok{(FinalGradeCEMS }\SpecialCharTok{\textasciitilde{}}\NormalTok{ TimeSpent }\SpecialCharTok{+}\NormalTok{ subject, }\AttributeTok{data =}\NormalTok{ d)}
\FunctionTok{summary}\NormalTok{(m1)}
\end{Highlighting}
\end{Shaded}

\begin{verbatim}
## 
## Call:
## lm(formula = FinalGradeCEMS ~ TimeSpent + subject, data = d)
## 
## Residuals:
##     Min      1Q  Median      3Q     Max 
## -70.378  -8.836   4.816  12.855  36.047 
## 
## Coefficients:
##                Estimate Std. Error t value Pr(>|t|)    
## (Intercept)  57.3931739  2.3382193  24.546  < 2e-16 ***
## TimeSpent     0.0071098  0.0006516  10.912  < 2e-16 ***
## subjectBioA  -1.5596482  3.6053075  -0.433    0.665    
## subjectFrScA 11.7306546  2.2143847   5.297 1.68e-07 ***
## subjectOcnA   1.0974545  2.5771474   0.426    0.670    
## subjectPhysA 16.0357213  3.0712923   5.221 2.50e-07 ***
## ---
## Signif. codes:  0 '***' 0.001 '**' 0.01 '*' 0.05 '.' 0.1 ' ' 1
## 
## Residual standard error: 19.8 on 567 degrees of freedom
##   (30 observations deleted due to missingness)
## Multiple R-squared:  0.213,  Adjusted R-squared:  0.2061 
## F-statistic: 30.69 on 5 and 567 DF,  p-value: < 2.2e-16
\end{verbatim}

\hypertarget{your-turn-6}{%
\paragraph{\texorpdfstring{{\textbf{Your Turn}}
\textbf{⤵}}{Your Turn ⤵}}\label{your-turn-6}}

Notice how above the variables are separated by a + symbol. Below, add
\emph{another} - a third - variable to the regression model.
Specifically, add a variable students' initial, self-reported interest
in science, \texttt{int} - and any other variable(s) you like! What do
you notice about the results? We're going to dive into this \emph{much}
more: if you have many questions now, you're in the right spot!

\begin{Shaded}
\begin{Highlighting}[]
\NormalTok{m2 }\OtherTok{\textless{}{-}} \FunctionTok{lm}\NormalTok{(FinalGradeCEMS }\SpecialCharTok{\textasciitilde{}}\NormalTok{ TimeSpent }\SpecialCharTok{+}\NormalTok{ subject, }\AttributeTok{data =}\NormalTok{ d)}
\FunctionTok{summary}\NormalTok{(m2)}
\end{Highlighting}
\end{Shaded}

\begin{verbatim}
## 
## Call:
## lm(formula = FinalGradeCEMS ~ TimeSpent + subject, data = d)
## 
## Residuals:
##     Min      1Q  Median      3Q     Max 
## -70.378  -8.836   4.816  12.855  36.047 
## 
## Coefficients:
##                Estimate Std. Error t value Pr(>|t|)    
## (Intercept)  57.3931739  2.3382193  24.546  < 2e-16 ***
## TimeSpent     0.0071098  0.0006516  10.912  < 2e-16 ***
## subjectBioA  -1.5596482  3.6053075  -0.433    0.665    
## subjectFrScA 11.7306546  2.2143847   5.297 1.68e-07 ***
## subjectOcnA   1.0974545  2.5771474   0.426    0.670    
## subjectPhysA 16.0357213  3.0712923   5.221 2.50e-07 ***
## ---
## Signif. codes:  0 '***' 0.001 '**' 0.01 '*' 0.05 '.' 0.1 ' ' 1
## 
## Residual standard error: 19.8 on 567 degrees of freedom
##   (30 observations deleted due to missingness)
## Multiple R-squared:  0.213,  Adjusted R-squared:  0.2061 
## F-statistic: 30.69 on 5 and 567 DF,  p-value: < 2.2e-16
\end{verbatim}

\hypertarget{communicate}{%
\subsection{5. COMMUNICATE}\label{communicate}}

Great job! Once you've finished your work, Upon doing so, you should see
a new \texttt{laser-orientation-badge.html}.

Congratulations, you've completed your Models \& Inference Badge!
Complete the following steps to submit your work for review by

\begin{enumerate}
\def\labelenumi{\arabic{enumi}.}
\item
  Change the name of the \texttt{author:} in the
  \href{https://monashdatafluency.github.io/r-rep-res/yaml-header.html}{YAML
  header} at the very top of this document to your name. As noted in
  \href{https://monashdatafluency.github.io/r-rep-res/index.html}{Reproducible
  Research in R}, The YAML header controls the style and feel for
  knitted document but doesn't actually display in the final output.
\item
  Click the yarn icon above to ``knit'' your data product to a
  \href{https://bookdown.org/yihui/rmarkdown/html-document.html}{HTML}
  file that will be saved in your R Project folder.
\item
  Commit your changes in GitHub Desktop and push them to your online
  GitHub repository.
\item
  Publish your HTML page the web using one of the following
  \href{https://rpubs.com/cathydatascience/518692}{publishing methods}:

  \begin{itemize}
  \item
    Publish on \href{https://rpubs.com}{RPubs} by clicking the
    ``Publish'' button located in the Viewer Pane when you knit your
    document. Note, you will need to quickly create a RPubs account.
  \item
    Publishing on GitHub using either
    \href{https://pages.github.com}{GitHub Pages} or the
    \href{http://htmlpreview.github.io}{HTML previewer}.
  \end{itemize}
\item
  Post a new discussion on GitHub to our
  \href{https://github.com/orgs/laser-institute/teams/foundations/discussions/2}{Foundations
  Badges forum}. In your post, include a link to your published web page
  and a short reflection highlighting one thing you learned from this
  lab and one thing you'd like to explore further.
\end{enumerate}

\end{document}
